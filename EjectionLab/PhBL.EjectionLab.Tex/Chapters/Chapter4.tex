\chapter{Заключение}

В процессе выполнения работы по представленной математической модели был выполнен итерационный расчет. На каждой итерации производился расчет геометрии струи, скорости эжектируемого воздуха и потерь. Изменения внешнего давления в приструйной зоне, возникающие в следствие эжекции, учитывались при каждой последующей итерации расчета. В момент, когда границы струи достигли стенок камеры (6 итерация), была также пересчитана площадь эжектирующего участка. После этого, было произведено еще 6 итераций.

В результате выполнения расчета выявлены зависимости, представленные графически в \hyperref[Chapter3]{главе 3}. Из них видно, что давление в приструйной зоне линейно падает \hyperref[fig:ExternalPressure]{(рис. 3.2.1)} в пределах от 998,2 кПа до 986,4 кПа. Данное обстоятельство приводит к расширению границ струи в среднем на 0.005 м по всех длине \hyperref[fig:FluidStructure]{(рис. 3.2.4)} и последующему их пересечению со стенками камеры.