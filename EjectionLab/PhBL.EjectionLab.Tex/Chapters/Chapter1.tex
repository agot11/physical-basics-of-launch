\chapter{Введение}

В работе рассматривается процесс эжекции воздуха из прилегающей к сверхзвуковой струе среды. Процесс истечения происходит в камеру ограниченного диаметра. В качестве результата работы необходимо определить закономерности изменения давления в следствие эжекции в приструйной зоне. При расчете необходимо учесть потери давления на эжекцию воздуха и потери на местные споротивления в эжектируемом потоке в зазоре между стенками камеры.